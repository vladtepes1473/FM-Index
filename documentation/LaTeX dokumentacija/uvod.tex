Porast količine tekstualnih informacija u zadnja dva desetljeća potaknuo je razvoj struktura podataka koje će tu informaciju memorijski efikasno pohraniti, omogućujući pritom vremensku efikasnost pretrage i dohvata. Jedno od područja koje ima veliku korist od razvoja ovakvih struktura je bioinformatika. Radi same prirode bioinformatičkih problema, odnosno, obrade bioloških podataka (npr. analiza dugačkih sekvenci DNK), vrlo je pogodno nad takvim biološkim podacima izgraditi spomenute strukture kako bi se njihova analiza ubrzala.
Jedan od pristupa rješavanja ovog problema je upotreba indeksa pretraživanja po cijelom tekstu (eng. full-text search). Ove strukture omogućuju brzo i potpuno pretraživanje teksta nad kojim su izgrađene. Nakon izgradnje, pronalazak proizvoljnog tekstualnog uzorka (zajedno s njegovim brojem ponavljanja) vrlo je učinkovit. No, ovaj pristup, koji stavlja naglasak na brz pronalazak traženih uzoraka, često ne ispunjava memorijske zahtjeve. Budući da korisnost pretrage raste s količinom teksta, memorijska efikasnost vrlo je bitan faktor pri odabiru načina obrade i pohrane teksta kojega je potrebno analizirati.
Novije izvedbe indeksa dobiveni tekst prvo sažmu, te tek nakon toga, nad sažetim tekstom grade indekse. Jedan od takvih indeksa je FM-index čija je implementacija i zadatak ovoga projekta.
