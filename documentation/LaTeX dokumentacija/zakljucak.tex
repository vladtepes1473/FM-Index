Iako je izgradnja FM-indeksa za veće ulazne datoteke zahtjevan i dugotrajan proces, prednost takvog pristupa je kasnije efikasno pretraživanje i dohvaćanje željenih tekstualnih uzoraka. Implementacijom opisanom u ovome radu, postignuta je izgradnja indeksa s linearnom složenošću, koja ovisi samo o veličini ulazne datoteke. Pokazalo se da je opisanom implementacijom postignuto to da vrijeme prebrojavanja nekog uzorka ne ovisi o veličini ulaznog niza nad kojim je indeks izgrađen, već o veličini abecede ulaznog niza. Što se tiče memorijske složenosti, prilikom izgradnje indeksa, ona iznosi otprilike 6n, pri čemu je n veličina ulaznog niza. Nakon izgradnje indeksa, zauzeta memorija se smanjuje, te tada ovisi o veličini abecede ulaznog niza. Unatoč nedostacima programskog jezika Java u kojem je napravljena implementacija, vremenska i memorijska složenost ostvarenog algoritma je zadovoljavajuća.